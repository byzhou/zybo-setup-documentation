% ==========================================
% Include packages
% I am not using all the packages, comment out the ones
% you don't need:
% \usepackage[utf8]{inputenc}
\usepackage{amsmath}	% for mathematical symbols
\usepackage[justification=centering]{caption}	% captions for graphics
\usepackage{colortbl}
\usepackage[firstpage]{draftwatermark}
\SetWatermarkText{Analog Devices Inc.}
%\usepackage{enumerate}	% different enumerate styles
\usepackage{enumitem}   % another enumerate package
\usepackage{fancyhdr}	% Nice headers/footers here
\usepackage{float}      % floats are important
\usepackage[bottom]{footmisc} % Footnotes at the bottom
\usepackage{gensymb}          % Generic symbols
\usepackage[textwidth=7in]{geometry}	% define paper geometry
\usepackage{graphicx}	% Graphic environment
\usepackage{hyperref}	% to use hyperlinks and hyper-references
\usepackage[all]{hypcap}% to use hyperlinks and hyper-references
% \usepackage{layout}	% This is just to show the layout of the page, comment out!
\usepackage{listings}	% coding environment (used to show verilog codes)
\usepackage{multirow}	% multirow tables, can use with the figures
% \usepackage{pstricks}   % PostScript Images
% \usepackage{subcaption}	% for subcaptions in multigraphics
\usepackage{subfigure}	% for subcaptions in multigraphics
\usepackage{tabularx}	% for use of tabularX environments
\usepackage{url}        % Sometimes you might want to refer to a URL  
\usepackage{verbatim}	% For verbatim text
\usepackage{wrapfig}	% to use text-wrapping around the figures
\usepackage{xcolor}	% using it to define code colors
\usepackage{algorithm}
\usepackage{../include/packages/algorithmicx/algpseudocode}


