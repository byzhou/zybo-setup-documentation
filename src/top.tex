%%%%%%%%%%%%%%%%%%%%%%%%%%%%%%%%%%%%%%%%%%%%
% You can use 
% \verbatiminput{<file_name>} 
% to show source codes. Alternatively, use 
% \verb+<your_code_here>+ to show code inline
% or use
% \begin{verbatim}
%    ...
% \end{verbatim}
%%%%%%%%%%%%%%%%%%%%%%%%%%%%%%%%%%%%%%%%%%%%
% ../include/code_params has a customized
% code styling (for Verilog), please, modify
% to use with any other source code.
% Also, ../include/custom_comms has necessary
% functions for the lstlistings
%%%%%%%%%%%%%%%%%%%%%%%%%%%%%%%%%%%%%%%%%%%%

%\documentclass[12pt,DIV12]{scrartcl}	% I find this document type the nicest, can use beavel
\documentclass[12pt]{article} % This document is the most neutral

% ==========================================
% Modify these on your own risk... :)
% ==========================================
% Include packages
% I am not using all the packages, comment out the ones
% you don't need:
% \usepackage[utf8]{inputenc}
\usepackage{amsmath}	% for mathematical symbols
\usepackage[justification=centering]{caption}	% captions for graphics
\usepackage{colortbl}
\usepackage[firstpage]{draftwatermark}
\SetWatermarkText{Boston University, Electrical and Computer Engineering}
%\usepackage{enumerate}	% different enumerate styles
\usepackage{enumitem}   % another enumerate package
\usepackage{fancyhdr}	% Nice headers/footers here
\usepackage{float}      % floats are important
\usepackage[bottom]{footmisc} % Footnotes at the bottom
\usepackage{gensymb}          % Generic symbols
\usepackage[textwidth=7in]{geometry}	% define paper geometry
\usepackage{graphicx}	% Graphic environment
\usepackage{hyperref}	% to use hyperlinks and hyper-references
\usepackage[all]{hypcap}% to use hyperlinks and hyper-references
% \usepackage{layout}	% This is just to show the layout of the page, comment out!
\usepackage{listings}	% coding environment (used to show verilog codes)
\usepackage{multirow}	% multirow tables, can use with the figures
% \usepackage{pstricks}   % PostScript Images
% \usepackage{subcaption}	% for subcaptions in multigraphics
\usepackage{subfigure}	% for subcaptions in multigraphics
\usepackage{tabularx}	% for use of tabularX environments
\usepackage{url}        % Sometimes you might want to refer to a URL  
\usepackage{verbatim}	% For verbatim text
\usepackage{wrapfig}	% to use text-wrapping around the figures
\usepackage{xcolor}	% using it to define code colors
\usepackage{algorithm}
\usepackage{../include/packages/algorithmicx/algpseudocode}


	% Packages
\input{ ../include/page_params} 	% Page layout and parameters
\input{ ../include/custom_comms}	% Custom commands
% ==========================================
% Define code styles:
\lstdefinestyle{custom_verilog}{
  basicstyle=\footnotesize\ttfamily,
  belowcaptionskip=0.1\baselineskip,
  breaklines=true,
  captionpos=t,
  commentstyle=\color{green!40!black},
  deletekeywords={...},
  % Note that @ sign is better to use as an escape character 
  % but in Verilog it is already reserved:
  escapeinside={^}{^},
  frame=L,
  identifierstyle=\color{black},
  keywordstyle=\bfseries\color{blue},
  language=Verilog,
  morecomment={
        [s][\color{red}]{/*!}{*}
  },
  morekeywords={
        *,
        % Data type definitions:
        bit,
        electrical,
        logic,
        longint,
        shortint,
        shortreal,
        typedef,
        wreal,
        ...,
        % extra constructs:
        interface,
        endinterface,
        exclude,
        from,
        ...,
        % blocks:
        analog,
        ...},
  numbers=left,
  numberstyle=\zebra{gray!50}{white}, % Don't forget to declare ZEBRA command 
  showstringspaces=false,
  stepnumber=1,
  stringstyle=\color{orange},
  tabsize=2,
  % title=\lstname,
  xleftmargin=\parindent,
}

% Define ADICE language:
\lstdefinelanguage{ADICE}{
  alsoletter = {\$},
  comment = [l]{\linebreak*},
  keywords = {
    *,
    command,
    eval, deval
    if, elseif, else, endif,
    include,
    keep,
    open, close,
    print, printf,
    radar,
    set, reset,
    use,
    while, endwhile,
  },
  sensitive = false,
  string = [b]",
  % MORE:
  % morecomment = [l]{**},
  % moredelim = [l]{(}{)},
}

% Change the code caption (Listing is kinda lame):
\renewcommand{\lstlistingname}{Code Listing}

\lstset{style=custom_verilog} % By default assume all codes are verilog type
 	% Code params for LISTINGS package


% ==========================================
% Document parameters
% You can include pictures here
\addHeader{BU}{ECE} % {Left}{Right}

% ==========================================
% Document information:
\title{Zybo Board Setup Documentation}
% \subtitle{Tutorial}
\date{2015 fall}
\author{ 
  Boston University\thanks{email \href{mailto:bobzhou@bu.edu}{bobzhou@bu.edu} 
    for the source codes}
}

% ==========================================
% Main Body:
\begin{document}
        \maketitle
        
        \section *{Motivation}
        \label{sec:motivation}
        Zybo board contains an ARM core and an FPGA for both GPP (General Purpose Processor)
        calculations and high-performance calculations with FPGA. The setup for the Zybo board is
        pretty awkward, since student does not have unlimited license for the board compiler,
        vivado. This documentation provides an convenient way in setting up compiling, synthesis and
        mapping tutorials for Zybo board. The purpose is that the board can eventually self-tunned
        to functional specific high-performance computing as GPP needed. 
        \pagebreak
        \tableofcontents

        \pagebreak
             
        \section{Introduction}
        \label{sec:introduction}

        \section{Installation}
        \label{sec:installation}
        
        \section{StaticIP}
        \label{sec:StaticIP}
        The static IP is for setting up the wire connection between two machines. In order to have static
IP, you need to turn off the DHCP. The way to turn off the DHCP and assign another static IP system
is to modify the following file.

\begin{lstlisting}
    /etc/network/interfaces
\end{lstlisting}

These few lines of codes need to be added in the \verb+interfaces+ file.

\begin{lstlisting}
    auto eth0
    iface eth0 inet static
    address 192.168.1.50
    netmask 255.255.255.0
    network 192.168.0.0
\end{lstlisting}

For the configuration of \textit{ssh} is in the following file, including the port setup.

\begin{lstlisting}
    /etc/ssh/sshd_config
\end{lstlisting}

For restart the eth0 port, the code is listed below. Type in these in command line.
\begin{lstlisting}
    sudo ifdown eth0 && sudo ifup eth0
\end{lstlisting}

For restart the network manager, type in the following commands in terminal.
\begin{lstlisting}
    sudo service network-manager restart
\end{lstlisting}


        \section{Setup Xillybus IP Interface}
        \label{sec:xillybusIP}
        This part is for setting up the xillybus IP environment. The xillybus is the IP of the interface on
the FPGA. The Xillybus company also provides the software level support, which are the interfaces on
the linux system. The software support includes the kernel module and the app software. App software
utilize the pthread for both read and write in order to increase the throughput of the FPGA API. But
first, we need to setup the SD card for the linux system.

The micro SD card needs to be formatted for the particular file system. Here is the tutorial of
formatting the micro SD card. Follow only the first three steps.

\begin{lstlisting}
    http://www.instructables.com/id/Setting-up-the-Zybot-Software/
\end{lstlisting}

The reformatting commands
\begin{lstlisting}
    sudo mkfs -t vfat -n ZYBO_BOOT /dev/sdb1
    sudo mkfs -t ext4 -L ROOT_FS /dev/sdb2
\end{lstlisting}


The following folder is downloaded from the Xillybus website, which includes all the hardware
related files.
\begin{lstlisting}
    xillinux-eval-zybo-1.3c
\end{lstlisting}

Inside of the folder, the bootfiles go to the SD card's ZYBO\_BOOT partition. The downloaded img file, 
\begin{lstlisting}
    xillinux-1.3.img
\end{lstlisting}
include linux file system and uImage. The uImage goes into the  ZYBO\_BOOT partition and linux file
system goes to ROOT\_FS.

The other files in xillinux-eval-zybo-1.3c are the examples of verilog codes. The example verilog
code are the APIs that are used in the FPGA application. The one that I like is in the folder
\begin{lstlisting}
    verilog
\end{lstlisting}
It is the verilog code for a fifo inside of FPGA that is passing all the information from the ARM
core back to ARM core.

Install the module 
\begin{lstlisting}
    make
    make install
    cp 10-xillybus.rules /etc/udev/rules.d/
    modprobe xillybus_pcie
    lsprobe
\end{lstlisting}

Basic testing 
\begin{lstlisting}
    date > /dev/xillybus_write_8
    cat /dev/xillybus_read_8
\end{lstlisting}



        \section{Conclusions}
        \label{sec:conclusion}

\begin{lstlisting}
    path\to\your\colorirun parameters ....
\end{lstlisting}
        
        \pagebreak
        \bibliographystyle{unsrt}
        \bibliography{../src/top}
        
\end{document}
