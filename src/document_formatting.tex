%%%%%%%%%%%%%%%%%%%%%%%%%%%%%%%%%%%%%%%%%%%%%%%%%%%%%%%%
% This document just shows the formatting as a reference

\subsection *{Document Formatting}
\label{subsec:doc_format}
\hrulefill \\
This section describes the formatting in the~\LaTeX~document
\begin{itemize}
\item \verb+Input to different forms and file names are in this font+ 
\item \verb+$ Terminal_input_has -a "$" -in front+ 
\item[] 
\begin{lstlisting}
always @ (*) begin
  The codes right now are tuned to Verilog;
  Modify the lstlisting to support other languages;
end
\end{lstlisting}
\item \textbf{Menu $\Rightarrow$ Navigation $\Rightarrow$ Has 
  $\Rightarrow$ arrows}
\item \textit{Window names} and \textit{``Input form names''} are in italic
\end{itemize}
\hrulefill



%        \section {Writing codes in the \LaTeX~environment}
%        You can write your codes using \verb+lstlisting+. The default style is set to 
%        \verb+custom_verilog+, which is a Verilog environment (defined in the 
%        \verb+./include/code_params.tex+). To use the default environement use:
%
%        \begin{lstlisting}[language={[LaTeX]{TeX}},numbers=none]
%\begin{^\verb+lstlisting+^}[label=code:my_label,caption={My Caption}]
%  <Some Code Here>;
%  <More Code Here>;
%  <etc.>
%\end{^\verb+lstlisting+^}
%        \end{lstlisting}
%        \verb+label+ keyword allows you to reference your code, while \verb+caption+ creates a
%        caption (Duh!).
%
%        \section {Special functions that you can use}
%        \label{sec:functions}
%        I have written some \LaTeX functions that you can use. All of them are defined in the
%        \verb+./include/+ \verb+custom_comms.tex+ file. The possible functions are:
%        \begin{lstlisting}[language={[LaTeX]{TeX}},numbers=none]
%\makespace % Creates spacing after the \wrapfigure
%        \end{lstlisting}
%        
%        \subsection {Functions for collaboration}
%        \label{sec:functions:collaboration}
%        These functions might help you if you are writing a document with someone:
%        \begin{lstlisting}[language={[LaTeX]{TeX}},numbers=none]
%\fixme{This needs to be fixed! Are you sure this is right?}
%        \end{lstlisting}
%        Result: \\
%        \fixme{This needs to be fixed! Are you sure this is right?}
%
%        \begin{lstlisting}[language={[LaTeX]{TeX}},numbers=none]
%I have written a long text, and I am not sure if whatever after that is important.
%\ignore{This text although was written will not be shown on the final document.}
%I have ignored it for now, to see how it would look like without it.
%        \end{lstlisting}
%        Result: \\
%        I have written a long text, and I am not sure if whatever after that is important.
%        \ignore{This text although was written will not be shown on the final document.}
%        I have ignored it for now, to see how it would look like without it.
%
\begin{algorithm}[H]
    \caption{UDT}
    \label{alg:resUDT}
    \begin{algorithmic}[1]
        \State \verb+Structure+ UDT: resNode
        \State Voltage
        \State Current
        \State Impedance \Comment{This is the impedance measured from a port.}
        \State Sum of Previous Stage's Impedance \Comment{This is for current calculation.}
        \State Port Direction \Comment{This should be a boolean attribute, but I used enumerate
        variable.}
    \end{algorithmic}
\end{algorithm}
