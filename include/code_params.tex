% ==========================================
% Define code styles:
\lstdefinestyle{custom_verilog}{
  basicstyle=\footnotesize\ttfamily,
  belowcaptionskip=0.1\baselineskip,
  breaklines=true,
  captionpos=t,
  commentstyle=\color{green!40!black},
  deletekeywords={...},
  % Note that @ sign is better to use as an escape character 
  % but in Verilog it is already reserved:
  escapeinside={^}{^},
  frame=L,
  identifierstyle=\color{black},
  keywordstyle=\bfseries\color{blue},
  language=Verilog,
  morecomment={
        [s][\color{red}]{/*!}{*}
  },
  morekeywords={
        *,
        % Data type definitions:
        bit,
        electrical,
        logic,
        longint,
        shortint,
        shortreal,
        typedef,
        wreal,
        ...,
        % extra constructs:
        interface,
        endinterface,
        exclude,
        from,
        ...,
        % blocks:
        analog,
        ...},
  numbers=left,
  numberstyle=\zebra{gray!50}{white}, % Don't forget to declare ZEBRA command 
  showstringspaces=false,
  stepnumber=1,
  stringstyle=\color{orange},
  tabsize=2,
  % title=\lstname,
  xleftmargin=\parindent,
}

% Define ADICE language:
\lstdefinelanguage{ADICE}{
  alsoletter = {\$},
  comment = [l]{\linebreak*},
  keywords = {
    *,
    command,
    eval, deval
    if, elseif, else, endif,
    include,
    keep,
    open, close,
    print, printf,
    radar,
    set, reset,
    use,
    while, endwhile,
  },
  sensitive = false,
  string = [b]",
  % MORE:
  % morecomment = [l]{**},
  % moredelim = [l]{(}{)},
}

% Change the code caption (Listing is kinda lame):
\renewcommand{\lstlistingname}{Code Listing}

\lstset{style=custom_verilog} % By default assume all codes are verilog type
