This part is for setting up the xillybus IP environment. The xillybus is the IP of the interface on
the FPGA. The Xillybus company also provides the software level support, which are the interfaces on
the linux system. The software support includes the kernel module and the app software. App software
utilize the pthread for both read and write in order to increase the throughput of the FPGA API. But
first, we need to setup the SD card for the linux system.

The micro SD card needs to be formatted for the particular file system. Here is the tutorial of
formatting the micro SD card. Follow only the first three steps.

\begin{lstlisting}
    http://www.instructables.com/id/Setting-up-the-Zybot-Software/
\end{lstlisting}

The reformatting commands
\begin{lstlisting}
    sudo mkfs -t vfat -n ZYBO_BOOT /dev/sdb1
    sudo mkfs -t ext4 -L ROOT_FS /dev/sdb2
\end{lstlisting}


The following folder is downloaded from the Xillybus website, which includes all the hardware
related files.
\begin{lstlisting}
    xillinux-eval-zybo-1.3c
\end{lstlisting}

Inside of the folder, the bootfiles go to the SD card's ZYBO\_BOOT partition. The downloaded img file, 
\begin{lstlisting}
    xillinux-1.3.img
\end{lstlisting}
include linux file system and uImage. The uImage goes into the  ZYBO\_BOOT partition and linux file
system goes to ROOT\_FS.

The other files in xillinux-eval-zybo-1.3c are the examples of verilog codes. The example verilog
code are the APIs that are used in the FPGA application. The one that I like is in the folder
\begin{lstlisting}
    verilog
\end{lstlisting}
It is the verilog code for a fifo inside of FPGA that is passing all the information from the ARM
core back to ARM core.

Install the module 
\begin{lstlisting}
    make
    make install
    cp 10-xillybus.rules /etc/udev/rules.d/
    modprobe xillybus_pcie
    lsprobe
\end{lstlisting}

Basic testing 
\begin{lstlisting}
    date > /dev/xillybus_write_8
    cat /dev/xillybus_read_8
\end{lstlisting}

